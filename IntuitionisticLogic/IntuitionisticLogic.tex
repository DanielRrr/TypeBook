\documentclass[a4paper]{article}
\usepackage{setspace}
\usepackage{amsmath}
\usepackage{pgfplots}
\usepackage[utf8]{inputenc}
\usepackage{tikz-cd}
\usepackage[all, 2cell]{xy}
\usepackage{amssymb}
\usepackage{verbatim}
\usepackage[all]{xy}
\usepackage{tikz}
\usetikzlibrary{graphs}
\usetikzlibrary{arrows}
\usepackage{hyperref}
\usepackage[english,russian]{babel}
\begin{document}
\title{Введение в интуиционистскую логику.}
\section{Критика классической математики Я. Брауэром и основные принципы интуиционистской математики.}
\section{Неформальная семантика Брауэра-Колмогорова-Гейтинга.}
\section{Интуиционистское исчисление высказываний и шкалы Крипке.}
\section{Элементы интуиционистской теории доказательств.}
\end{document}
