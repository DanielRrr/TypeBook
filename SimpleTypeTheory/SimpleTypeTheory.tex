\documentclass[a4paper]{article}
\usepackage{setspace}
\usepackage{amsmath}
\usepackage{pgfplots}
\usepackage[utf8]{inputenc}
\usepackage{tikz-cd}
\usepackage[all, 2cell]{xy}
\usepackage{amssymb}
\usepackage{verbatim}
\usepackage[all]{xy}
\usepackage{tikz}
\usetikzlibrary{graphs}
\usetikzlibrary{arrows}
\usepackage{hyperref}
\usepackage[english,russian]{babel}
\begin{document}
\title{Лямбда-исчисление.}
\section{Немного истории.}
\section{Введение в лямбда-исчисление: определения и базовые результаты.}

Рассмотрим мотивирующий пример. Когда мы пишем, что <<функция отображает аргумент $x$ в $M$>>, где $M$ --- это метапеременная,
в которой лежит тело функции, то мы используем следующую нотацию $x \mapsto M$, тогда как запись $\lambda x.M$ следует читать точно
также в содержательном смысле. Расширим наш мотивирующий и не совсем формальный пример,
заменив метапеременную $M$ на более понятное арифметическое выражение : $x \mapsto x^2 + 6x + 9$ и $\lambda x.x^2 + 6x + 9$.

Теперь же перейдем к формальным определениям. Базовое понятие в $\lambda$-исчислении --- это \emph{предтерм}. Предположим, у нас есть бесконечный алфавит:

\begin{equation}
\Lambda = {v_0, v_1, v_2, v_3, ... }
\end{equation}

\emph{Предтермами} мы будем называть конечные строки над алфавитом $\Lambda$, порожденные следующей грамматикой:

\begin{equation}
\Lambda_{term} ::= \Lambda \: | \: (\Lambda_{term} \Lambda_{term}) \: | \: \lambda \Lambda . \Lambda_{term}
\end{equation}

Примеры конечных строк, порожденных заданной грамматикой:

1) $((v_3 \: v_5) \: v_8)$;

2) $\lambda v_6. v_5 v_6$

3) $\lambda v_0. v_0$

4) $\lambda v_{05091995}.\lambda v_{38}.v_{4}$


Как мы видим из определения грамматики, предтермы бывают трех видов.

Зададим классификацию предтермов в соответсвии с грамматикой:

1) Предтерм первого вида (это просто элементы $\Lambda$) называется \emph{переменной}, которые мы будем обозначать тремя предпоследними буквами
латинского алфавита $x, y, z, ...$ (возможно с индексами);

2) Предтерм второго вида (записанные два подряд предтерма) называется \emph{аппликацией} (или \emph{применением}), которую мы будем обозначать $(M N)$, где
$M$ и $N$ --- это произвольные предтермы, которые впредь будут обозначаться метапеременными $M, N, O,...$ (возможно с индексами);

3) Предтерм третьего вида (знак $\lambda$ с переменной, точка и предтерм) называется $\lambda$-\emph{абстракцией}, которая будет обозначаться как $\lambda x.M$,
где $x$ является \emph{связанной переменной}. Если в предтерме встречается переменная $x$, которая связана $\lambda$-оператором,
то такая переменная будет называться \emph{свободной переменной}.

Поясним, что $\lambda$ --- это оператор связывания. Пусть у нас есть некоторый предтерм $M$, содержащий свободные вхождения $x$. Теперь мы $\lambda$-абстрагируемся
по $x$ и получим предтерм третьего вида $\lambda x. M$, предъявляя таким образом выражение, зависящее от значения параметра $x$.

\textbf{Важное терминологическое соглашение}: любой предтерм, удовлетворяющий тому или иному виду, мы будет называть $\lambda$-термами.

\section{Комбинаторная логика и ее связь с лямбда-исчислением.}
\section{Простое типизированное лямбда-исчисление: типизация по Карри и по Черчу.}
\section{Типизированные комбинаторы.}
\section{Практическая реализация лямбда-исчисления, комбинаторной логики и теории типов.}
\end{document}
