\documentclass[a4paper]{article}
\usepackage{setspace}
\usepackage{amsmath}
\usepackage{pgfplots}
\usepackage[utf8]{inputenc}
\usepackage{tikz-cd}
\usepackage[all, 2cell]{xy}
\usepackage{amssymb}
\usepackage{verbatim}
\usepackage[all]{xy}
\usepackage{tikz}
\usetikzlibrary{graphs}
\usetikzlibrary{arrows}
\usepackage{hyperref}
\usepackage[english,russian]{babel}
\begin{document}
\title{Лямбда-исчисление.}
\section{Немного истории.}
\section{Введение в лямбда-исчисление: определения и базовые результаты.}

Рассмотрим мотивирующий пример. Когда мы пишем, что <<функция отображает аргумент $x$ в $M$>>, где $M$ --- это метапеременная,
в которой лежит тело функции, то мы используем следующую нотацию $x \mapsto M$, тогда как запись $\lambda x.M$ следует читать точно
также в содержательном смысле. Расширим наш мотивирующий и не совсем формальный пример,
заменив метапеременную $M$ на более понятное арифметическое выражение : $x \mapsto x^2 + 6x + 9$ и $\lambda x.x^2 + 6x + 9$.

Теперь же перейдем к формальным определениям. Базовое понятие в $\lambda$-исчислении --- это \emph{предтерм}. Предположим, у нас есть бесконечный алфавит:

\begin{equation}
\Lambda = {v_1, v_2, v_3, ... }
\end{equation}

\emph{Предтермами} мы будем называть конечные строки над алфавитом $\Lambda$, порожденные следующей грамматикой:


\begin{equation}
\Lambda_{term} ::= \Lambda \: | \: \Lambda \Lambda \: | \: \lambda \Lambda . \Lambda  
\end{equation}

\section{Комбинаторная логика и ее связь с лямбда-исчислением.}
\section{Простое типизированное лямбда-исчисление: типизация по Карри и по Черчу.}
\section{Типизированные комбинаторы.}
\section{Практическая реализация лямбда-исчисления, комбинаторной логики и теории типов.}
\end{document}
