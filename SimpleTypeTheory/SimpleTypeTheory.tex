\documentclass[a4paper]{article}
\usepackage{setspace}
\usepackage{amsmath}
\usepackage{pgfplots}
\usepackage[utf8]{inputenc}
\usepackage{tikz-cd}
\usepackage[all, 2cell]{xy}
\usepackage{amssymb}
\usepackage{verbatim}
\usepackage[all]{xy}
\usepackage{tikz}
\usetikzlibrary{graphs}
\usetikzlibrary{arrows}
\usepackage{hyperref}
\usepackage[english,russian]{babel}
\begin{document}
\title{Лямбда-исчисление.}
\section{Немного истории.}
\section{Введение в лямбда-исчисление: определения и базовые результаты.}
\section{Комбинаторная логика и ее связь с лямбда-исчислением.}
\section{Простое типизированное лямбда-исчисление: типизация по Карри и по Черчу.}
\section{Типизированные комбинаторы.}
\section{Практическая реализация лямбда-исчисления, комбинаторной логики и теории типов.}
\end{document}
